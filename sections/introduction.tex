\section{Introduction}
Due to the fact, that a lot of monetary transactions happen via credit cards, the potential for fraudulent actions in this domain is increasing. 
Since these fraudulent actions result in great loss of money for credit card owners or companies, this topic has become a priority and many data scientists have started focussing their research on fiscal fraud detection.
While detection of fraudulent credit card actions may happen in real-time, fiscal fraud detection in tax offices is offline\footnote{According to \cite{gruhl2022}, an algorithm works in an offline fashion, if it requires access to the complete data set at once to perform a successful detection.}. Although tax evasion does not directly affect individuals, but the government, techniques of credit card fraud detection may be adapted to identify tax evasion. 
The goal of this survey is to outline four of the ways researchers have already found to identify financial fraud with unsupervised techniques.

The remainder of this article is structured as follows. 
First, the problem at hand is discussed in \autoref{sec:problem}. After that in \autoref{sec:state_of_art}, the existing techniques for a transferable anomaly detection problem are briefly listed. In \autoref{sec:main_part}, four unsupervised techniques are described in more detail. Afterwards, these techniques are compared in a discussion about their potential with regard to the problem at hand. Finally, there is a conclusion and outlook.
