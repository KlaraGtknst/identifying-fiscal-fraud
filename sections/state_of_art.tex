
\section{State of the art}
\label{sec:state_of_art}
In this section, technical terms specific to the domain are defined and briefly discussed.
Moreover, the structure of feasible credit card data is outlined.
In the end, a brief overview of some of the financial fraud detection techniques is provided.

\subsubsection{Financial fraud:}
Financial fraud may be defined by \textquote[from \cite{ff_review_techniques}, based on \cite{acfe_ff}]{any intentional act to deprive another of property or money by guile, deception, or other unfair means}.
There are several different types of financial fraud, for instance, money laundering, credit card or insurance fraud as listed in \cite{ff_review_techniques}.
This survey's focus is on credit card fraud, since its detection techniques can be transferred to the problem presented in \autoref{sec:problem}.
According to \cite{ff_review_techniques} there are multiple types of credit card fraud including application fraud\footnote{identity fraud to issue new cards from credit companies} and behavioural fraud (unauthorised usage of credit cards by third parties, bankruptcy fraud\footnote{filing for personal bankruptcy instead of paying back the balance}).


\subsubsection{Anomalies:}
There is no general definition of an \textit{anomaly}. However, the authors of \cite{ff_review_techniques} use the definition of Hawkins, which identifies outliers or anomalies as observations which deviate so much from the other observations as to arouse suspicions that they were generated by a different mechanism. There are three different types of anomalies, which are examined in \cite{gruhl2022, ff_review_techniques}.

\textit{Point anomalies} are single instances of the data, which deviate from the rest of the data set. They can occur in any data set.

There are \textit{contextual} and \textit{behavioural} attributes. For instance, \texttt{time} in time-series data describes the context of that data and is therefore a \textit{contextual} attribute. Regarding this example from \cite{ff_review_techniques}, the attribute \texttt{temperature} is a non-contextual attribute and thus, is a \textit{behavioural} attribute.
\textit{Contextual anomalies} are individual instances which deviate in terms of the \textit{behavioural} attribute in a specific context. For instance, the temperature of a certain month is exorbitant greater than the ones documented in the respective months in other years. These anomalies can only occur if there are contextual attributes in the data.

Moreover, Gruhl discusses \textit{collective} anomalies or so-called \textit{novelties}, which are multiple data instances, whose occurrence as a group is anomalous with respect to the whole data set, in \cite{gruhl2022}. They can only occur if there is a relationship between the data instances.


\subsubsection{Challenges of Anomaly Detection:}
Anomaly detection is challenging due to a variety of reasons pointed out in \cite{gruhl2022, ff_review_techniques}.
Firstly, there are numerous types of normal behaviourism and therefore, it is difficult to consider all of them when classifying an individual instance. Respectively, anomalous individuals may differ greatly in their characteristics as they are irregular.
Moreover, the boundary between normal and anomalous individuals often lacks precision. 
It is difficult to determine, since the underlying model of normal data may change over time.
Another issue when dealing with anomaly detection tasks is the lack of labelled data for training as a consequence of the costly and time-consuming nature of this task.
Given that there is usually some degree of natural noise in the data, it is difficult to distinguish between noise and actual anomalies.
One also has to consider changing the data set's dimension. However, certain relations or anomalous behaviours may be evident in a certain dimension but become hidden after a dimension reduction.


\subsubsection{Data labels:}
The goal of anomaly detection is to assign a label to every instance of a data set indicating whether it is an anomaly or not.
In some cases, there are already labels present for a portion of the data set.
As stated in \cite{ff_review_techniques}, there are different types of anomaly detection techniques which either require labelled data (semi-/supervised) or not (unsupervised).


\subsubsection{Credit card data:}
Credit card data may be stored as a file containing $n$ features, such as \texttt{time}, \texttt{amount}, \texttt{id}, for $m$ transactions. 
Each transaction $t_i$ has a unique id $i$.
The data sets are highly imbalanced.
There is data available online, for instance on kaggle, however, due to confidentiality issues, the data from \cite{kaggle_ex} is not provided in its original form.
In this case, \acfi{PCA} transformation was used on all features except \texttt{time} and \texttt{amount}.
Originally, credit card data may contain non-numeric features.
Depending on the context, credit card data contains transactions of one or multiple individuals.
Hence, in some cases, the data is considered a time series.


\subsubsection{Models and techniques:}
Data preprocessing is important, since, for instance (supervised) machine learning techniques ignore minority classes in imbalanced data sets and thus, perform poorly on them.

According to \cite{ff_profiles}, cardholder profiles are created based on the cardholders' spending behaviours and patterns in order to detect credit card fraud. Abnormal transactions, such as drawing out unusually large amounts of money from inconvenient locations, may indicate fraudulent actions.

As stated in \cite{ff_review_techniques}, models used for identifying credit card fraud include \textit{\aclp{DT}} (\acs{DT})\acused{DT}, \acfi{SVM}, \acfi{LR} and \acfi{kNN}, which are outlined in \cite{sahin2010detecting}. It is also possible to combine those models in an ensemble along the lines of the \acfi{RF} algorithm.
Frequently used deep learning anomaly detection techniques and architectures are \acfi{NN}, \acfi{CNN}, \acfi{LSTM}, \textit{\aclp{AE}} (\acs{AE})\acused{AE} and \acfi{GAN}.
The models and techniques stated above are outlined in \cite{ff_review_techniques}.

The unsupervised financial fraud detection techniques \acfi{IF}, \acfi{AE}, \acfi{ARIMA}, \acfi{SOM}, which form the basis of this survey and are respectively described in \cite{liu2008isolation, cf_AE, fd_ARIMA, fd_SOM, credit_f_SOM}.